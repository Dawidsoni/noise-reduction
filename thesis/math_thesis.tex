\documentclass[a4paper, 11pt, onecolumn, openany, titlepage]{report}
\usepackage[utf8]{inputenc}
\usepackage{enumitem}
\usepackage{filecontents}
\usepackage[T1]{fontenc}
\usepackage{url}
\usepackage{breakcites}
\usepackage{graphicx}
\graphicspath{{qualitative-results/}{quantitative-results/}}
\usepackage{caption}
\usepackage{subcaption}
\usepackage{titlesec}
\usepackage{tabularx}
\usepackage{afterpage}
\usepackage{geometry}
\usepackage{hyperref}
\usepackage[rightcaption]{sidecap}
\usepackage[usenames, dvipsnames]{xcolor}
\usepackage[nottoc]{tocbibind}

\geometry{a4paper, headsep=1.0cm, footskip=1cm, lmargin=3cm, rmargin=3cm, tmargin=3cm, bmargin=3cm}
\hypersetup{colorlinks, citecolor=NavyBlue, filecolor=NavyBlue, linkcolor=NavyBlue, urlcolor=NavyBlue}

\titleformat{\chapter}[block]{\normalfont\huge\bfseries}{\thechapter.}{5pt}{\huge}
\titlespacing*{\chapter}{0pt}{-19pt}{25pt}
\titleformat{\section}[block]{\normalfont\Large\bfseries}{\thesection.}{5pt}{\Large}

\newcommand\blankpage{\null\thispagestyle{empty}\newpage}
\newcommand\numberedchapter[1]{\setlength\topskip{3cm}\chapter{#1}\setlength\topskip{0cm}}
\newcommand\unnumberedchapter[1]{\setlength\topskip{3cm}\chapter*{#1}\setlength\topskip{0cm}}

\begin{document}
\setlength\topskip{3cm}
\newpage
\thispagestyle{empty}
\begin{center}
\textbf{\large Uniwersytet Wrocławski\\
Wydział Matematyki i Informatyki\\
Instytut Matematyczny}\\
\vspace{4cm}
\textbf{\textit{\large Dawid Wegner}\\
\vspace{0.5cm}
{\Large Efektywne odszumianie obrazów przy użyciu algorytmu Metropolis-Hastings}}\\
\end{center}
\vspace{3cm}
{\large \hspace*{6.5cm}Praca licencjacka\\
\hspace*{6.5cm}napisana pod kierunkiem\\
\hspace*{6.5cm}dr hab. Pawła Lorka}\\
\vfill
\begin{center}
{\large Wrocław 2021}\\
\end{center}
\setlength\topskip{0cm}
\afterpage{\blankpage}

{\hypersetup{linkcolor=black}
\setlength\topskip{3cm}
\tableofcontents
\setlength\topskip{0cm}
}

\unnumberedchapter{Introduction}
\addcontentsline{toc}{chapter}{Introduction}

TODO


\numberedchapter{Markov Chains}

TODO

\numberedchapter{Markov chain Monte Carlo methods}

TODO

\section{Gibbs sampling}

TODO

\section{Metropolis–Hastings algorithm}

TODO

\numberedchapter{Ising model and its applications}

TODO

\section{Formal definition of the Ising model}

TODO

\section{Application to binary image denoising problem}

TODO

\section{Extension to grayscale images}

TODO

\section{Applying MCMC methods to the Ising model}

TODO

\numberedchapter{Gradient-based image denoising}

TODO

\section{Proposal distribution inspired by gradients}

TODO

\section{Efficient implementation for image denoising problem}

TODO

\numberedchapter{Experiments: binary images}

TODO

\section{Methodology}

TODO

\section{Denoising quality based on image size}

TODO

\section{Denoising quality based on noise level}

TODO

\section{Sensitivity of noise level prior parameter}

TODO

\section{Qualitative results}

TODO

\numberedchapter{Experiments: grayscale images}

TODO

\section{Methodology}

TODO

\section{Denoising quality}

TODO

\section{Qualitative results}

TODO

\numberedchapter{Summary}

TODO

Podstawowym problemem, z jakim zmagają się wszyscy przedsiębiorcy, jest przygotowa- nie oferty w ten sposób, by była ona zarówno atrakcyjna dla klienta (a przynajmniej bardziej atrakcyjna niż to, co oferuje konkurencja), jak i przynosiła zyski dla przedsiębiorcy (a już na pewno nie generowała strat). Branża ubezpieczeniowa nie jest tu wyjątkiem. Każda ubez- pieczalnia zatrudnia sztab aktuariuszy, których zadaniem jest wyznaczenie obowiązkowych składek w ten.
Computers \cite{einstein} sdsd.
Podstawowym problemem, z jakim zmagają się wszyscy przedsiębiorcy, jest przygotowa- nie oferty w ten sposób, by była ona zarówno atrakcyjna dla klienta (a przynajmniej bardziej atrakcyjna niż to, co oferuje konkurencja), jak i przynosiła zyski dla przedsiębiorcy (a już na pewno nie generowała strat). Branża ubezpieczeniowa nie jest tu wyjątkiem. Każda ubez- pieczalnia zatrudnia sztab aktuariuszy, których zadaniem jest wyznaczenie obowiązkowych składek w ten sposób, by pokryły koszta działalności firmy oraz przyszłe zobowiązania wzglę- dem klientów. W przypadku ubezpieczeń majątkowych zadanie jest o tyle trudniejsze, że bardzo istotny jest tak zwany czynnik ludzki czyli np. skłonność do ryzyka lub umiejętność prowadzenia pojazdów. Są to parametry trudne do zmierzenia, stąd potrzeba jak najlepsze- go modelowania wpływu tych składowych na wartość zgłaszanych szkód, na skutek których konieczna będzie wypłata odszkodowania. Obecnie najczęściej stosowaną metodą jest sys- tem Bonus - Malus, którego głównym założeniem jest związanie wysokości składki z liczbą wypadków spowodowanych przez kierowcę. Wadą tego rozwiązania jest nieuwzględnianie istot- nych czynników takich jak np. częstotliwość prowadzenia pojazdu czy styl jazdy. Pomysłem na rozwiązanie tego problemu jest montowanie w samochodach urządzeń rejestrujących jaz- dę, jednak może się to spotkać z oporem klientów dbających o ochronę swojej prywatności. Powstaje więc pytanie jak poprawić stosowane rozwiązania bez konieczności zbierania dodat- kowych danych o kierowcy. Odpowiedzią na to pytanie mogą być tak zwane ukryte modele Markowa.
Figure \ref{fig:xxx} compared with Figure \ref{fig:yyy} sss \cite{einstein} sdsd.


\begin{figure}
\centering
\includegraphics[scale=0.4]{input_size_plots}
\caption{Example of a parametric plot ($\sin (x), \cos(x), x$)}
\label{fig:xxx}
\end{figure}
\begin{figure}
\centering
\includegraphics[scale=0.4]{input_size_plots}
\caption{Example of a parametric plot ($\sin (x), \cos(x), x$)}
\label{fig:yyy}
\end{figure}

\begin{thebibliography}{9}
\bibitem{latexcompanion}
Michel Goossens, Frank Mittelbach, and Alexander Samarin.
\textit{The \LaTeX\ Companion}.
Addison-Wesley, Reading, Massachusetts, 1993.

\bibitem{einstein}
Albert Einstein.
\textit{Zur Elektrodynamik bewegter K{\"o}rper}. (German)
[\textit{On the electrodynamics of moving bodies}].
Annalen der Physik, 322(10):891–921, 1905.

\bibitem{knuthwebsite}
Knuth: Computers and Typesetting,
\\\texttt{http://www-cs-faculty.stanford.edu/\~{}uno/abcde.html}
\end{thebibliography}


\end{document}
